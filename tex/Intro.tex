\section{Introduction to Extend}
	Extend is a domain-specific programming language used to designate ranges of cells as reusable functions. It abstracts dependencies between cells and models a dependency graph during compilation. In order to offer great performance for any size of datasets, Extend compiles down to LLVM.
	
	Extend's syntax is meant to provide clear punctuation and easily understandable cell range access specifications, while borrowing elements from languages with C-style syntax for ease of development. Despite these syntactic similarities, the semantics of an Extend program have more in common with a spreadsheet such as Microsoft Excel than imperative languages such as C, Java or Python.
\section{Structure of an Extend Program}
\label{sec:Grammar}
	An Extend program consists of one or more source files. A source file consists of an optional set of import directives, an optional set of global variable declarations, and an optional set of function declarations, in that order. The grammar for a source file is displayed below:
\lstinputlisting{./src/main/parser.mly}
	\subsection{Import Statements}
		Import statements in Extend are written with \texttt{import}, followed by the name of a file in double quotes, and terminated with a semicolon. The syntax is as follows:
		\begin{lstlisting}
import "string.xtnd";
		\end{lstlisting}
		
		Extend imports act like \texttt{\#include} in C, except that multiple imports of the same file are ignored, and Extend rearranges the initial import and global statements to properly compile. The imports are all aggregated into a single namespace.
	\subsection{Global Variables}
		In essence, global variable declarations function as constants in Extend. They are written with the keyword \texttt{global}, followed by a variable declaration in the same form as a variable declaration within a function as described in section~\ref{sec:vardecl}. 
	\subsection{Function Declarations}
		Function declarations are described in detail in section~\ref{sec:Functions}.
\section{Types and Literals}
	\subsection{Primitive Data Types}
		Extend has two primitive data types, \textbf{numbers} and \textbf{empty}. In the vein of Javascript, numbers are primitive values corresponding to a double-precision 64-bit binary format IEEE 754 value. Numbers can be written in an Extend source file as either integer or floating point constants; both are represented internally as floating-point values. The \textbf{empty} type can be written as the keyword \texttt{empty} or the empty string \texttt{""}, and serves a similar function to \texttt{NULL} in SQL.
		\newline
		\begin{table}[H]
		\centering
		\begin{tabular} {| l | l |}
			\hline
			\textbf{Primitives} & \textbf{Examples} \\ \hline
			Number & \texttt{42 or -5 or 2.71828 or 314159e-5} \\ \hline
			Empty & \texttt{empty or ""} \\ \hline
		\end{tabular}
		\end{table}
	\subsection{Ranges}
		Extend has one composite type, the \textbf{range}. A range borrows conceptually from spreadsheets; it is a group of cells with two dimensions, described as rows and columns. Each cell contains a formula that either evaluates to a number or another range. Cell formulas are described in detail in section~\ref{sec:formula}. A range can either be declared as described in section~\ref{sec:vardecl} or with a range literal expression.
\subsubsection{Range and String Literals}
		A range literal is a semicolon-delimited list of rows, enclosed in curly brackets. Each row is a comma-delimited list of numbers or ranges. In addition, a range literal can be written in the form of a string literal, which represents a 1-by-n range corresponding to the ASCII values of the contents of the string. A few examples follow: 
		
\begin{lstlisting}
{2,3,4} /* A range consisting of 1 row and 3 columns */
{1,0,0;0,1,0;0,0,1} /* A range corresponding to the 3x3 identity matrix */
"hello" /* Equivalent to {104,101,108,108,111} */
{"hello";"world"} /* A range with 2 rows and 1 column; 
both cells of the range contain a range */
\end{lstlisting}
