\section{Types and Literals}
		Extend has three primitive data types, \textbf{Number}, \textbf{String}, and \textbf{Empty}, and one composite type, \textbf{Range}.
	\subsection{Primitive Data Types}
		A \textbf{Number} is an immutable primitive value corresponding to a double-precision 64-bit binary format IEEE 754 value. Numbers can be written in an Extend source file as either integer or floating point constants; both are represented internally as floating-point values. There is no separate type representing an integer.

		A \textbf{String} is a immutable primitive value that is internally represented a C-style null-terminated byte array corresponding to ASCII values. A String can be written in an Extend source file as a sequence of characters enclosed in double quotes, with the usual escaping conventions. Extend does not allow for slicing of strings to access specific characters; access to the contents of a string will only be available through standard library functions.

		The \textbf{Empty} type can be written as the keyword \texttt{empty}, and serves a similar function to \texttt{NULL} in SQL; it represents the absence of a value.
		\newline
		\begin{table}[H]
		\centering
		\begin{tabular} {| l | l |}
			\hline
			\textbf{Primitive Data Types} & \textbf{Examples} \\ \hline
			Number & \texttt{42 or -5 or 2.71828 or 314159e-5} \\ \hline
			String & \texttt{"Hello, World!\textbackslash n" or "foo" or ""} \\ \hline
			Empty & \texttt{empty} \\ \hline
		\end{tabular}
		\end{table}
	\subsection{Ranges}
		Extend has one composite type, \textbf{Range}. A range borrows conceptually from spreadsheets; it is a group of cells with two dimensions, described as rows and columns. Each cell is assigned a formula that either evaluates to a Number, a String, \texttt{empty}, or another Range. Cell formulas are described in detail in section~\ref{sec:formula}. A range can either be declared as described in section~\ref{sec:vardecl} or with a range literal expression. Ranges can be nested arbitrarily deeply and can be used to represent (immutable) lists, matrices, or more complicated data structures.
\subsubsection{Range Literals}
		A range literal is a semicolon-delimited list of rows, enclosed in curly brackets. Each row is a comma-delimited list of numbers, strings, or range literals. A few examples follow:
		\lstinputlisting{./samples/legal_range_literals.xtnd}
