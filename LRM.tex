\documentclass[titlepage]{article}

\usepackage[english]{babel}
\usepackage[utf8]{inputenc}
\usepackage{amsmath}
\usepackage{amsthm}
\usepackage{graphicx}
\usepackage[colorinlistoftodos]{todonotes}
\usepackage{amssymb}
\usepackage{mathtools}
\usepackage{listings}
\usepackage[inline]{enumitem}
\usepackage{tikz}
\usepackage{tabularx}
\usepackage{lstautogobble}
\usepackage{algorithm}
\usepackage{todonotes}
\usepackage[noend]{algpseudocode}
\usepackage[a4paper,left=3cm,right=3cm,top=3cm,bottom=3cm]{geometry} 
\usepackage{titlesec}
\usepackage{siunitx}
\usepackage{mathrsfs}
\usepackage{setspace}
\usepackage{fixltx2e}
\usepackage{hyperref}
\usepackage{csquotes}
\usepackage{colortbl}
\linespread{2}
\titlelabel{\thetitle.\quad}
\allowdisplaybreaks
\delimitershortfall-1sp

\usetikzlibrary{arrows}
\usetikzlibrary{automata,positioning}
\lstset{basicstyle=\ttfamily,
  mathescape=true,
  escapeinside=||,
  autogobble}
  
\title{Extend\\ Language Reference Manual}
\author{Ishaan Kolluri, Kevin Ye, Jared Samet, Nigel Schuster}
\date{\today}
\begin{document}
\maketitle
\tableofcontents
\section{Introduction to Extend}
Extend is a domain-specific programming language used to designate ranges of cells as reusable functions. It abstracts dependencies between cells and builds a dependency graph during compilation. Extend compiles to LLVM in order to take advantage of its extensive garbage collection features.
\newline
Extend's syntax is meant to provide clear punctuation and specificity for cell ranges, while maintaining the look of modern functional programming languages. Given Extend's functionality resonates well with spreadsheets, it borrows syntactical elements from programs such as Microsoft Excel.
\section{Types and Literals}
	\subsection{Primitive Data Types}
		Extend's basic primitives are \textit{integers}, \textit{floats}, \textit{char} literals, and \textit{string} literals. They are all internally represented as numbers or a range of numbers. They are as follows:
		\newline
		A \textbf{char} literal is essentially a size 1 numerical range. At evaluation, the number in the range will be compared with its ASCII equivalent.
  		\newline
  		A \textbf{string} literal is a range of numbers of size \textit{n}, where \textit{n} is the length of the string. The string 'hello' can be represented internally as \{42,23,18,27,-3\}.
		A \textbf{integer} can be represented as a size 1 numerical range as well. However, it retains its numerical value upon evaluation. 
		\newline
		\textbf{TODO: Get a definition for float }
		\newline
		Below is a snippet illustrating declarations for each of the above types.
  		\begin{lstlisting}
			/* Integer */
  			num = 5;
			/* Char */
			chr = 'A'
			/* String */
			str = 'Hello'
  		\end{lstlisting}
\subsection{Ranges}
\subsubsection{Range Slicing}
\subsubsection{The Underscore Symbol}
\subsection{Integers and Strings}
\subsubsection{Unary Operations}
\section{Expressions}
\subsection{Operators}
\subsubsection{Multiplication}
\subsubsection{Addition}
\subsubsection{Unary}
\subsection{Booleans}
\subsection{Variable Declaration}
\subsection{Variable Assignment}
\subsection{Conditionals}
\section{Functions}
\subsection{Format}
\subsection{Dimension Assignment}
\subsection{Application across Dimensions}
\subsection{Dependencies Illustrated}
\section{I/O}
\subsection{File I/O}
\subsection{Input Arguments}
\subsubsection{How to run a program}
\subsubsection{Main function}
\section{Example Program}
\end{document}