\section{Expressions}
	Expressions in Extend allows for arithmetic and boolean operations, function calls, conditional branching, and extraction of contents of other variables. In Section 2 “Structure of an Extend Program,” it shows the OCAML grammar for Extend and exactly what an expression can be under “expr:”. In this section, Extend expressions are broken down into the following 5 categories: Operators, Booleans, Variable Declaration, Variable Assignment, and Conditionals.


	\subsection{Operators}
		Extend has three different operator types: \textbf{Multiplication}, \textbf{Addition}, and \textbf{Unary}. The precedence of the operators are listed in order from the highest starting in section 4.1.1. and each subsequent subsection is of lower precedence than the previous subsection. 
		\newline

		\subsubsection{Multiplication, Quotient, and Modulo}
			The multiplication, division, and modulo operators are of equal and highest precedence when it comes to operations. The expressions for all three operators are arithmetic types.
			\newline
			\underline{\textbf{Multiplication}}\newline
			The binary '*' operator computes the \textbf{multiplication} of two expressions.
			\begin{lstlisting}
Syntax:
expr * expr 
			\end{lstlisting}
			\underline{\textbf{Quotient}}\newline
			The binary '/' operator computes the \textbf{quotient} of two expressions. The quotient operator will perform the quotient as a decimal operation. This means that if you assign both expressions are of type integer, then the result will be an integer. For example, the quotient of 5/2 will compute to 2 and not 2.5. This can be solved by making one or both expressions to type float. Another thing to note is that if the second expression or divisor is 0, then it will yield an undefined value.
			\begin{lstlisting}
Syntax:
expr / expr 
			\end{lstlisting}
			\underline{\textbf{Modulo}}\newline
			The binary '\%' operator computes the \textbf{modulo} of two expressions. The modulo operator finds the remainder from dividing the expression to the left of the modulo symbol, the dividend, with the expression to the right of the modulo symbol, the divisor. For the modulo operator to return a correct value, the expression to the left of the modulo symbol or dividend must be greater than 0.
			\begin{lstlisting}
Syntax:
expr % expr 
			\end{lstlisting}
		\subsubsection{Addition and Subtraction}
			The Addition and Subtraction operators are of equal and second highest precedence after Multiplication, Division, and Modulo when it comes to operations. The expressions for all three operators are also arithmetic types.
			\newline
			\underline{\textbf{Addition}}\newline
			The binary operator ‘+’ computes the addition of two expressions. 
			\begin{lstlisting}
Syntax:
expr + expr 
			\end{lstlisting}
			\underline{\textbf{Subtraction}}\newline
			The binary operator ‘-’ computes the subtraction of two expressions. 
			\begin{lstlisting}
Syntax:
expr - expr 
			\end{lstlisting}
		\subsubsection{Unary}
			The unary operators operate on a single expression. Extend uses three unary operators: Minus, Logical Complement, and Bitwise Complement.
			\newline
			\underline{\textbf{Minus}}\newline
			The unary Minus operator '-' can be used to negate numbers of type integer, decimal and floating-point.
			\begin{lstlisting}
Syntax:
-expr 
			\end{lstlisting}
			\underline{\textbf{Logical Complement}}\newline
			The unary Logical Complement Operator '!', also known as negation, can be used to negate a boolean type.
			\begin{lstlisting}
Syntax:
!expr 
			\end{lstlisting}
			\underline{\textbf{Bitwise Complement}}\newline
			The unary Bitwise Complement Operator '\textasciitilde' can be used to to get the bitwise complement of an expression. This operator can only be used on an integral type. 
			\begin{lstlisting}
Syntax:
~expr 
			\end{lstlisting}


	\subsection{Booleans}
		Extend supports boolean comparisons between two expressions. The operators that compare two expressions and return a boolean value are Equal To, Not Equal To, Greater Than, Less Than, Greater Than Or Equal To, Less Than Or Equal To, And, and Or. 
		\begin{lstlisting}
/* this checks if the two expressions are equal     */
expr == expr 
/* this checks if the two expressions are not equal */
expr != expr
/* this checks if the first expression is greater   */
/* than the second expression                       */
expr > expr
/* this checks if the first expression is less than */
/* the second expression                            */
expr < expr
/* this checks if the first expression is greater   */
/* than or equal to the second expression           */
expr >= expr
/* this checks if the first expression is less than */
/* or equal to the second expression                */
expr <= expr
/* this checks if the first expression is true AND  */
/* if the second is true                            */
expr && expr
/* this checks if the first expression is true OR  */
/* if the second is true                            */
expr || expr
		\end{lstlisting}
		

	\subsection{Conditionals}
		Conditionals statements and conditional expressions allow you to control your program to do different actions in different situations. Extend uses the operators Switch, Case, and Default to do this. A Switch statement evaluates an expression, checks each Case statement for a match and executes the statements in that Case statement. If none of these cases match, then it will execute the Default case. An example of this is shown below:
		\begin{lstlisting}
[1,1] foo = 3;
return switch {
	case foo = 2:
		"foo is 2";
	case foo = 3:
		"foo is 3";
	case foo = 4:
		"foo is 4";
	default:
		"foo is none of the cases"
}
		\end{lstlisting}