\section{Functions}
\label{sec:Functions}
The bulk of an Extend program consists of functions. Although Extend has some features, such as immutability and lazy evaluation, that are inspired by functional languages, its functions are not \textit{first class objects}. Although the standard library is, by default, automatically compiled and linked with a program, there are no functions built into the language itself.
\subsection{Format}
\label{sec:funcdecl}
As in most programming languages, the header of the function declares the parameters it accepts. The body of the function consists of an optional set of variable declarations and formula assignments, which can occur in any order, and a return statement, which must be the last statement in the function body. All variable declarations and formula assignments, in addition to the return statement, must be terminated by a semicolon.
This very simple function returns whatever value is passed into it:
\lstinputlisting{./samples/functions_simple.xtnd}
\subsection{Variable Declarations}
\label{sec:vardecl}
A variable declaration associates an identifier with a range of cells of the specified dimensions, which are listed in square brackets before the identifier. For convenience, if the square brackets and dimensions are omitted, the identifier will be associated with a single cell. In addition, multiple identifiers, separated by commas, can be listed after the dimensions; all of these identifiers will be separate ranges, but with equal dimension sizes. The dimensions can be specified as any valid expression that evaluates to a Number, which will be rounded to the nearest signed 32-bit integer. If either dimension is zero or negative, or if the expression does not evaluate to a Number, a runtime error causing the program to halt will occur.
\begin{lstlisting}
[2, 5] foo; // Declares foo as a range with 2 rows and 5 columns
[m, n] bar; // Declares bar as a range with m rows and n columns
[3, 3] ham, eggs, spam; // Declares ham, eggs and spam as distinct 3x3 ranges
baz; // Declares baz as a single cell
\end{lstlisting}
\subsection{Formula Assignment}
\label{sec:formula}
A formula assignment assigns an expression to a subset of the cells of a variable. Unlike most imperative languages, this expression is not immediately evaluated, but is instead only evaluated if and when it is needed to calculate the return value of the function. A formula assignment consists of an identifier, an optional pair of slices enclosed in square brackets specifying the subset of the cells that the assignment applies to, an \texttt{=}, and an expression, followed by a semicolon. As with the expressions specifying the dimensions of a range, these slices specifying the cell subset can contain arbitrary expressions, as long as the expression taken as a whole evaluates to a Number, which will be rounded to the nearest signed 32-bit integer. Negative numbers are legal in these slices, and correspond to (dimension length + value).
\begin{lstlisting}
[5, 2] foo, bar, baz; // Declares foo, bar, and baz as distinct 5x2 ranges
foo[0,0] = 42; // Assigns the expression 42 to the first cell of the first row of foo
foo[0,1] = foo[0,0] * 2; // Assigns (foo[0,0] * 2) to the 2nd cell of the 1st row of foo
bar = 3.14159; // Assigns pi to every cell of every row of bar
baz[1:-1,0:1] = 2.71828; // Assigns e to cells (1,0) through (3,1), inclusive, of baz

/* The next line assigns foo[[-1],0] + 1 to every cell in
   both columns of foo, besides the first row */
foo[1:,:] = foo[[-1],0] + 2;
\end{lstlisting}
The last line of the source snippet above demonstrates the idiomatic Extend way of simulating an imperative language's loop; foo[4,0] would evaluate to 42+2+2+2+2 = 50 and foo[4,1] would evaluate to (42*2)+2+2+2+2 = 92.
\subsubsection{Combined Variable Declaration and Formula Assignment}
For convenience, a variable declaration and a formula assignment to all cells of that variable can be combined on a single line by inserting a \texttt{:=} and an expression after the identifier. Multiple variables and assignments, separated by commas, can be declared on a single line as well. All global variables must be defined using the combined declaration and formula assignment syntax.
\begin{lstlisting}
/* Creates two 2x2 ranges; every cell of foo evaluates to 1 and every cell of
   bar evaluates to 2. */
[2,2] foo := 1, bar := 2;
\end{lstlisting}
\subsubsection{Formula Assignment Errors}
If the developer writes code in such a way that more than one formula applies to a cell, this causes a compile-time error if the compiler can detect it or a runtime error if the compiler cannot detect it in advance and the cell is evaluated. If there is no formula assigned to a cell, the cell will evaluate to \texttt{empty}.
\subsection{Parameter Declarations}
Parameters can be declared with or without dimensions. If dimensions are declared, they can either be specified as integer literals or as identifiers. If a dimension is specified as an integer literal, the program will verify the dimension of the argument before beginning to evaluate the return expression; if it does not match, a runtime error will occur causing the program to halt. If it is specified as an identifier, that variable will contain the dimension size and will be available inside the function body. If the same identifier is repeated in the function declaration, the program will verify that every parameter dimension with that identifier has equal dimension size; if they differ, a runtime error will occur causing the program to halt. A few examples follow:
\lstinputlisting{./samples/param_decl_examples.xtnd}
\subsection{Application on Ranges}
Extend gives the developer the power to easily apply operations in a functional style on ranges. For example, the following function performs cell wise addition:
\lstinputlisting{./samples/functions_range_single.xtnd}
This function normalizes a column vector to have unit norm:
\lstinputlisting{./samples/functions_range_mix.xtnd}
\subsection{Lazy Evaluation and Circular References}
All cell values and variable dimensions are evaluated lazily if and when they are needed to calculate the return expression. Using lazy evaluation ensures that the cell values are calculated in a valid topological sort order and allows for detection of circular references; internally this is accomplished by constructing a function for each formula which is called if the cell's value is needed, and marking the cell as "in-progress" once it starts being evaluated and as "complete" once the value has been calculated. The only guarantees the language places on the order of cell evaluation are: (1) It will be a valid topological ordering; (2) In conditional expressions and in short-circuiting operator expressions, only the relevant conditional branches will be evaluated; and (3) In an expression using the precedence operator, the preceeding expression will be evaluated before the succeeding expression. A range selection consisting of multiple cells will not cause all the constituent cells to be evaluated; however, selection of a single cell will cause that cell's value to be evaluated.
If a program is written in such a way as to cause a circular dependency of one cell on another, and the return expression is dependent on that cell's value, a runtime error will occur. For example, in the following function:
\lstinputlisting{./samples/maybe_circular.xtnd}
A runtime error will occur if maybeCircular(1) is called; but if maybeCircular(0) is called, the function will simply return 0.
\subsection{External Libraries}
\label{sec:ExternFunctionSignatures}
Using the following library declaration:
\lstinputlisting{./samples/extern.xtnd}
will make the functions foo (taking two arguments) and bar (taking zero arguments) available within Extend. In LLVM, the compiler will declare external functions extend\textunderscore foo and extend\textunderscore bar as functions of two and zero arguments respectively. All arguments must have the type subrange\textunderscore p, and the function must have return type value\textunderscore p, both declared in the Extend standard library header file. In other words, the C file compiled to generate the library must have defined:
\begin{lstlisting}
value_p extend_foo(subrange_p arg1, subrange_p arg2) {
  /* function body here; */
}

value_p extend_bar() {
  /* function body here; */
}
\end{lstlisting}
