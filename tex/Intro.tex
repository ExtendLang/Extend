\section{Introduction to Extend}
	Extend is a domain-specific programming language used to designate ranges of cells as reusable functions. It abstracts dependencies between cells and models a dependency graph during compilation. In order to offer great performance for any size of datasets, Extend compiles down to LLVM.
	
	Extend's syntax is meant to provide clear punctuation and easily understandable cell range access specifications, while using C-style syntax for ease of development. Despite these syntactic similarities, the semantics of an Extend program have more in common with a spreadsheet such as Microsoft Excel than imperative languages such as C, Java or Python.
\section{Structure of an Extend Program}
	An Extend program consists of one or more source files. A source file consists of an optional set of import directives, an optional set of global variable declarations, and an optional set of function declarations, in that order. The grammar for a source file is displayed below:
\lstinputlisting{./src/main/parser.mly}
	\subsection{Import Statements}
		Import statements in Extend are written with \texttt{import}, followed by the name of a file in double quotes, and terminated with a semicolon. The syntax is as follows:
		\begin{lstlisting}
import "string";
		\end{lstlisting}
		
		Extend imports act like \texttt{\#include} in C, except that multiple imports of the same file are ignored, and Extend rearranges the initial import and global statements to properly compile. The imports are all aggregated into a single namespace.
	\subsection{Global Variables}
		In essence, global variable declarations function as constants in Extend. It is in the same form as a variable declaration within a function. See Section 5, Functions, for more information.
	\subsection{Function Declarations}
		Function declarations will additionally be described in more detail in Section 5.
\section{Types and Literals}
	\subsection{Primitive Data Types}
		Extend has two primitive data types, \textbf{numbers} and \textbf{empty}. In the vein of Javascript, numbers are primitive values corresponding to a double-precision 64-bit binary format IEEE 754 value. The \textbf{range}, Extend's composite data type, is elaborated on in the next section.
		Both \textbf{integer} and \textbf{float} can be represented with numbers. The \textbf{empty} type can be written as the keyword \texttt{empty} or the empty string \texttt{""}, and serves a similar function to \texttt{NULL} in SQL.
		\newline
		\begin{table}[H]
		\centering
		\begin{tabular} {| l | l |}
			\hline
			\textbf{Primitives} & \textbf{Examples} \\ \hline
			Integer & \texttt{5 or -5} \\ \hline
			Floating Point & \texttt{2.67 or -3.50} \\ \hline
			Empty & \texttt{empty or ""} \\ \hline
		\end{tabular}
		\end{table}
	\subsection{Ranges}
		Ranges are a data type unique to the Extend language. It borrows conceptually from spreadsheets; a range is a group of cells with two dimensions, represented as rows and columns. The range is composed of cells, and cells contain a formula that either evaluates to a number or another range. If a the range that a cell contains ultimately results in a circular dependency, a runtime exception will be thrown. A range is written as follows:
		\begin{lstlisting}
/* This is a left-handed range, used to assign a value. */
[1,2]foo; /*Range with 1 row and 2 columns */
		\end{lstlisting}
	\subsubsection{Range and String Literals}
		Ranges can be represented in literal form as well. Range literals can be written as \texttt{\{2, 3, 4\}}, which is a 1x3 range. String literals can be written simply as \texttt{"hello"}, which should evaluate to a 1x5 range that is internally represented as the numeric range \texttt{[104, 101, 108, 108, 111]}.