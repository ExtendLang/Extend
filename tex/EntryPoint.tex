\section{Entry Point, External Libraries, Scoping and Namespace}
\subsection{\texttt{main} function}
When a compiled Extend program is executed, the \texttt{main} function is evaluated. All computations necessary to calculate the return value of the function are performed, after which the program terminates. The \texttt{main} function must be a function of a single argument, conventionally denoted args, which is guaranteed to be a 1-by-n range containing the command line arguments.
\subsection{Scoping and Namespace}
For functions and for global variables, there is a single namespace that is shared between all files composing an Extend program, and they are visible throughout the entire program. As a result, multiple definitions of functions with the same name, whether in the same file or not, will cause a compile-time error. Functions declared in external libraries share the same namespace. For a local variable, the scope is the entire function in which it is defined. Functions may declare local variables sharing a name with a global variable; the local variable definition will override the global variable within that function.
\subsection{External Libraries}
\label{sec:ExternFunctionSignatures}
Using the following library declaration:
\lstinputlisting{./samples/extern.xtnd}
will make the functions foo (taking two arguments) and bar (taking zero arguments) available within Extend. In LLVM, the compiler will declare external functions extend\textunderscore foo and extend\textunderscore bar as functions of two and zero arguments respectively. All arguments must have the type subrange\textunderscore p, and the function must have return type value\textunderscore p, both declared in the Extend standard library header file. In other words, the C file compiled to generate the library must have defined:
\begin{lstlisting}
value_p extend_foo(subrange_p arg1, subrange_p arg2) {
  /* function body here; */
}

value_p extend_bar() {
  /* function body here; */
}
\end{lstlisting}
