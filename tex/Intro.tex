\section{Introduction to Extend}
	Extend is a domain-specific programming language used to designate ranges of cells as reusable functions. It is a dynamically-typed, statically-scoped, declarative language that uses lazy evaluation to carry out computations. Once computed, all values are immutable. In order to offer the best performance, Extend compiles down to LLVM.

	Extend's syntax is meant to provide clear punctuation and easily understandable cell range access specifications, while borrowing elements from languages with C-style syntax for ease of development. Despite these syntactic similarities, the semantics of an Extend program have more in common with a spreadsheet such as Microsoft Excel than imperative languages such as C, Java or Python.

\section{Structure of an Extend Program}
\label{sec:Grammar}
	An Extend program consists of one or more source files. A source file can contain any number of import directives, function definitions, global variable declarations, and external library declarations, in any order.
	\subsection{Import Statements}
		Import statements in Extend are written with \texttt{import}, followed by the name of a file in double quotes, and terminated with a semicolon. The syntax is as follows:
		\begin{lstlisting}
import "string.xtnd";
		\end{lstlisting}
		Extend imports act like \texttt{\#include} in C, except that multiple imports of the same file are ignored. The imports are all aggregated into a single namespace.
		\subsection{Function Definitions}
			Function definitions comprise the bulk of an Extend program. In short, a function consists of a set of variable declarations, formula assignments, and a return expression. Each variable consists of cells; the values of each cell are, if necessary, calculated according to formulas which each apply to a specified subset of the cells. Each cell value, once calculated, is immutable. A couple examples follow for context; functions are described in detail in section~\ref{sec:Functions}.
			\lstinputlisting{./samples/sum_column.xtnd}
	\subsection{Global Variables}
		In essence, global variable declarations function as constants in Extend. They are written with the keyword \texttt{global}, followed by a variable declaration in the combined variable declaration and assignment format described in section~\ref{sec:CombinedDeclAsgn}. As with local variables, the cell values of a global variable, once computed, are immutable. A few examples follow:
		\lstinputlisting{./samples/global.xtnd}
	\subsection{External Library Declarations}
	  An external library is declared with the \texttt{extern} keyword, followed by the name of an object file in double quotes, followed by a semicolon-delimited list of external function declarations enclosed by curly braces. A library declaration informs the compiler of the functions' names and signatures and instructs the compiler to link the
		object file when producing an executable. An external function declared as \texttt{foo} will call an appropriately written C function \texttt{\detokenize{extend_foo}}. An example follows:
		\lstinputlisting{./samples/extern.xtnd}
		This declaration would cause the compiler to link \texttt{mylib.o} and would make the C functions \texttt{\detokenize{extend_foo}} and \texttt{\detokenize{extend_bar}} available to Extend programs as \texttt{foo} and \texttt{bar} respectively. The required signature and format of the external functions is specified precisely in section~\ref{sec:ExternFunctionSignatures}.
		\subsection{\texttt{main} function}
		When a compiled Extend program is executed, the \texttt{main} function is evaluated. All computations necessary to calculate the return value of the function are performed, after which the program terminates. The \texttt{main} function must be a function of a single argument, conventionally denoted \texttt{args}, which is guaranteed to be a 1-by-n range containing the command line arguments.
		\subsection{Scoping and Namespace}
		For functions and for global variables, there is a single namespace that is shared between all files composing an Extend program, and they are visible throughout the entire program. Functions declared in external libraries share this namespace as well. For a local variable, the scope is the entire body of the function in which it is defined. Functions may declare local variables sharing a name with a global variable; inside that function, the name will refer to the local variable.
\lstinputlisting{./samples/scope_examples.xtnd}